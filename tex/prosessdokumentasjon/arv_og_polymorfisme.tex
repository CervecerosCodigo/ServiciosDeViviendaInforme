\section{Arv og polymorfisme}
I prosjektet ble det benyttet arv og polymorfisme i stor utstrekning med tanke på å gjenbruke så mye kode som mulig. Dette følte vi var krav for å lage et produkt med robust kode der alle moduler er separert fra hverandre. Samtidig har man veldig god kontroll på hva som må endres eller utbedres ved videre utvikling. Mye av komponentene vi bruker har utspring i én, eller veldig få superklasser, og ved endring av disse kan vi forandre store deler av programmet på en enkel måte.

Både personobjekter og boligobjekter har flere subklasser. Det samme har tabellens tabellmodell. Her brukes \texttt{instanceof} for å sjekke hvilken type objekter man har med å gjøre.

\subsection{Swing componenter}
Prinsippet for arv er i størst utstrekking brukt ved GUI-komponenter. De komponentene som ble oftest gjenbrukt ble redefinert i form av abstrakte klasser eller laget som "<Custom">-komponenter. Eksempel på dette er \texttt{JPanel, JTextField} eller \texttt{JButton}. Ved å la disse custom-komponentene arve de opprinnelige komponentene fra \texttt{Swing} kunne vi implementere en custom-komponent i stedet, og dermed raskt raskt endre alle de tilsvarende komponentene utseende og funksjon fra en sentral plass i programstrukturen, insteden for \textit{"<refactoring">} av komponentens definisjon over programmets alle filer.
Omtrent ingen komponenter brukes direkte fra \texttt{Swing} uten først å ha blitt tilpasset for gjenbruk flere steder.

\subsection{Pakkefordeling}
Programmet ble grundig delt opp i forskjellige pakker med tanke på å gruppere tilhørende komponenter sammen, og adskille dem fra komponenter som ikke logisk sett hører sammen. Dette er gjort spesielt med tanke MVC arkitektur. For eksempel alle klasser som bygger opp vinduer er plassert i \texttt{view}-pakke, deretter alle klasser som bygger opp registrerings vinduer er plassert \texttt{view.register}-pakken. For å gjenspeile denne analogien på controller-siden har vi lagt dem respektivt i \texttt{controller}-pakken og \texttt{controller.register}-pakken. Deretter ble alle klasser som \texttt{Person} eller \texttt{Bolig} lagt i pakken \texttt{model}. For grundig beskrivelse av fordeling av pakker og filer, se produktdokumentasjon avsnitt \ref{sec:klasserogpakker} side \pageref{sec:klasserogpakker}.

\subsection{Konstanter og enum}
Konstanter ble brukt med tanke på å ha en sentral definisjon av data som feks. konfigurasjoner. En sentral plassering av konstanter som \texttt{RegEx} eller dimensjoner for \texttt{swing}-komponenter gir mulighet til å sette og hente definisjoner til komponentene vi bruker fra en sentral plass. Enum-typer brukes spesifikt for å sette faste definisjoner angående et spesielt objekt. Et godt eksempel på dette er \textit{sivilstatus} der vi fyller opp en comboboks med enum-verdier i stedet for strenger. 