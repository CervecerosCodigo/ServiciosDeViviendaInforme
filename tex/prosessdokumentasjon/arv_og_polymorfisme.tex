\section{Arv og polymorfisme}
I prosjektet ble det benyttet arv og polymorfisme i stor utstrekning med hensikt på å gjenbruke så mye kode som mulig. Dette var et krav for å lage et produkt med robust kode der alle moduler er separert fra hverandre. Samtidig har man veldig god kontroll på hva som må endres eller utbedres ved videre utvikling. En slipper å gjøre samme endring på tvers av hele programmet.

Både personobjekter og boligobjekter har flere subklasser. Det samme ha tabellens tabellmodeller. Her brukes \texttt{instanceof} for å sjekke hvilken type objekter man har med å gjøre.

\subsection{Swing componenter}
Prinsippet for arv er i størst utstrekking brukt ved GUI-komponenter. De komponentene som ble oftest gjenbrukt ble redefinert i form av abstrakte klasser eller laget som "Custom"-komponenter. Eksempel på dette er \texttt{JPanel, JTextField} og \texttt{JButton}. Definering av abstrakte klasser av disse komponenter ga oss mulighet til å raskt endre komponentenes utseende og funksjon fra en sentral plass i programstrukturen istedet for \textit{"<refactoring">} av komponentens definisjon over hele programmet.\\  
Omtrent ingen komponenter brukes direkte fra \texttt{Swing} uten først å ha blitt tilpasset for gjenbruk flere steder.

\subsection{Pakkefordeling}
Programmet ble grundig delt opp i forskjellige pakker med tanke på å gruppere tilhørende komponenter sammen, og adskille dem fra komponenter som ikke logisk sett hører sammen. Dette gjort spesielt med tanke MVC arkitektur. For eksempel alle klasser som bygger opp vinduer er plassert i \emph{view}-pakke, deretter alle klasser som bygger opp registrerings vinduer er plassert \emph{view.register}-pakken. For å gjenspeile denne analogien på controller-siden har vi lagt dem respektivt i \emph{controller}-pakken og \emph{controller.register}-pakken. Deretter ble alle klasser som \emph{Person} eller \emph{Bolig} lagt i pakken \emph{model}. For grundig beskrivelse av fordeling av pakker og filer, se produktdokumentasjon avsnitt \ref{klasserogpakker} side \pageref{sec:klasserogpakker}.

\subsection{Konstanter og enum}
Konstanter ble brukt med tanke på å ha en sentral definisjon av data som feks. konfigurasjoner. En sentral plassering av konstanter som \texttt{RegEx} eller dimensjoner for \texttt{swing}-komponenter gir mulighet til å sette og hente definisjoner på en sentral plass. Enum-typer brukes spesifikkt for å sette faste definisjoner angående et spesielt objekt. Et godt eksempel på dette er \textit{sivilstatus} som brukes for å fylle opp combobokser ikke med tekst men med enum-type som kan sendes via kontroller til objekt for en direkte registrering. Dette er en mer praktisk metode enn kontrollering og bearbeiding av feks. strenger.    