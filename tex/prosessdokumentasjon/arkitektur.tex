\section{Arkitektur}
Programmet arkitektur bygger helt på \textit{Model View Controller} filosofi. Detter medfører at all logikk og grafiske brukegrensesnitt er helt separiert fra hverandre i egne moduler og pakker. All kommunikasjon mellom disse håndteres gjennom dedikerte kontrollere og interface. Denne arkitekturen ble valgt tidlig i prosjektet med hensikt på at holde koden så robust som mulig, slik tilnærmig gav oss mulighet til feks. kun inlkudere swing komponenter og kode som tilhører oprettelse av vinduer i egene klasse. All logikk som håndterer funksjoner til grafisk grensesnitt for et enkelt vindu finnes i en egen klasse. Dersom MVC ikke ble brukt for prosjektet hadde det resultert i meget store klasser mest sannsynlig som består av flere tusen rader per vindu. MVC gav oss mulighet til å rask legge til nye funksjoner til en alerede eksisterende vinduklasse. Et godt eksepel på dette er dersom samme vindu skal brukes både for registrering og editering av samme opplysninger. MVC stiler kun krav på at kontrolelren har to kontruktører, en som setter opp vinduet dersom det skal foretas en registrering og en konstruktør som kan brukes for endringer. Denne vil da oprette vindu og hente inn all nødvendig data fra objektet (model). 