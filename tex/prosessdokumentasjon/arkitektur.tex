\section{Arkitektur}
Programmets arkitektur bygger på \textit{Model View Controller} filosofi. Dette medfører at alle data og grafiske brukergrensesnitt er helt separert fra hverandre i egne moduler og pakker. All kommunikasjon mellom disse håndteres gjennom dedikerte kontrollere og \texttt{interface}. Denne arkitekturen ble valgt tidlig i prosjektet med tanke på å holde koden så robust som mulig. En slik tilnærming ga gjorde at de klassene som har med brukergrensesnittet å gjøre bare har den type komponenter, og ikke noe annet. Alt en trenger i brukergrensesnittet er \texttt{get-metoder} som gjør at kontrollerne kan kommunisere med komponentene der.

All logikk som håndterer funksjoner til grafisk grensesnitt for et enkelt vindu finnes i en egen klasse. Dersom MVC ikke hadde vært brukt i prosjektet hadde det resultert i meget store klasser, mest sannsynlig som består av flere tusen rader per vindu i noen tilfeller. \texttt{MVC} gav oss mulighet til å raskt legge til nye funksjoner til en allerede eksisterende vinduklasse. Et godt eksempel på dette er dersom samme vindu skal brukes både for registrering og editering av samme opplysninger. \texttt{MVC}-løsningen stiler kun krav om at kontrolleren har to konstruktører, én som setter opp vinduet dersom det skal foretas en nyregistrering og én konstruktør som kan brukes for endringer. Denne vil da opprette vindu og hente inn all nødvendig data fra objektet (model). 

Løsningen vi har kommet frem til har ytterligere gjenbruk av kode da det er samme de kontrollere som ligger bak både "meglersarkfanen" og "annonsearkfanen". I det man oppretter kontrollerne, sender man med hvilket vindu kontrolleren gjelder for. Man vil dermed ha én fysisk java-fil som styrer logikken til to vinduer som har tilnærmet lik funksjonalitet.