\section{Utfordringer}
Under veis i prosjektet har vi støtt på mange utfordringer. Hvis vi skal skrive ned alle så kan det resultere i at listen blir lengre enn selve rapporten. Vi har derfor valgt til å avgrense oss til de mest frekvente og de som gav oss størst utfordringer.

\subsection{Kontrollerhierarki}
I MVC-modellen, kan kommunikasjonen mellom kontroller og brukergrensesnitt håndteres av en metode i GUI-klassen som tar i mot en lytter som en parameter fra kontrolleren.
Situasjonen blir litt annerledes dersom man bruker kontrollere som ligger parallelt eller i en hierarki med hverandre uten å ha en naturlig måte for dem å kommunisere med hverandre. Den utfordringen har vi løst ved bruk av "lytter-interface" som implementeres av alle disse klassene. Dersom lytter-interface "<går av"> vil det bli oppfattet av alle som som har implementert interfacet. Utfordringen med den løsningen at koden blir fort ganske kompleks men vinsten blir god fleksibilitet.


\subsection{Serialisering}
De utfrodringer som vi støtte på ved serialsiering av objektene var at i begynnelsen ble det prøvd løsninger der vi serialiserte registrene i en klasse spesifikk laget for dette. 
Med det menes at vi lagde en klasse som kunde ta imot våre \texttt{collections} (dataregistere) via sin kontruktør og deretter serialisere dem. Data ble serialisert som tenkt men vi fikk ikke til å lese inn dataene igjen for å gjenopprette objektene på nytt. Den eneste løsningen som fungerte var at vi serialiserte alle registere (\texttt{HashSet}) og static variabler i samme klasse der de ble instansiert. Dette er en vel fungerende løsning men er ikke noe som vil være robust nok dersom man ønsker å øke programmets funksjonalitet. 


\subsection{Tabellmodell}
Det vi begynte på prosjektet hadde vi ikke lært mye om \texttt{JTable}. Da dette dukket opp på forelesning hadde vi allerede brukt det en stund, og funnet ut at vanlig \texttt{Array} ikke holder mål. Oracle sin dokumentasjon nevner bruk av \texttt{Array} og \texttt{Vectorer} som datakilde, men vi valgte å gå over til \texttt{ArrayList}. Vi har ett objekt per linje, og trengte ikke å bruke multidimensjonel implementasjon.
Første forsøk på å lage egen tabellmodell gikk tilsynelatende bra, men ved fjerning av elementer fra tabellen oppstod det enormt mange Null Pointer- og ArrayOutOfBound-exceptions. Spesielt om tabellen ble tømt for objekter, da hang hele programmet seg. Det tok en hel dag å bli kvitt dette problemet. 

\subsection{\texttt{Layout}-managere}
Under prosjektets gang har vi prøvd alle \texttt{Layout}-managere i \texttt{AWT} klassen.
Vi har hatt mange forskjellige behov underveis, i de forskjellige registreringsvinduene og panelene, og har endt opp med en god løsning der vi stort sett bruker \texttt{BorderLayout} på selve vinduet, og så en variasjon av \texttt{FlowLayout} og \texttt{GridBagLayout} innvendig i panelene til \texttt{BorderLayouten}.
I de vinduene vi bare har ett panel har vi brukt \texttt{GridLayout} med én rad og én kolonne. \\
Alle våre paneler arver enten \emph{AbstractPanel.java} eller er instans av \emph{CustomSubPanel.java} (som igjen arver \texttt{AbstractPanel.java} og begge de klassene tar i mot \texttt{Layout}-managere i minst én av sine konstruktører.

\subsection{Konflikter i GIT}
Ingen i gruppen hadde vært borte i GIT noe spesielt før prosjektet startet og spesielt de første to ukene brukte vi like mye tid på å lære oss dette og fikse konflikter, som vi brukte på å programmere. Da to av oss bruker Linux og Netbeans og én Mac med Eclipse, så har det vært en del utfordringer knyttet til dette også. Men å ta i bruk et vesrjonhånderingssystem som GIT har vært en god læring og gitt oss ny kunnskap som vi kan ta med oss videre til andre prosjekt under utdannelsen og ut i arbeidslivet.


\subsection{Objektreferanser}
I klassen \texttt{FreeTextSearch.java} har vi støtt på problemer med testing av objektype som skulle brukes i klassens søkemetoder. Ettersom vi bruker arv av Person for de forskjellige instansene av person som Utleier, Megler eller Leietaker ønsker man at det skal være mulig å teste på hvilken type av person som blir hentet fra registeret. 
For søking bruker vi generiske metoder, disse kan håndtere alle mulige register av type \texttt{HashSet}, ikke bare de som består av type Person. 
Ettersom vi vet hvilken type av set som vi kommer til å sende inn i metoden så hadde det vært optimalt å sende med en parameter som beskriver hvilken type av person som det skal søkes på. Se eksempel \ref{kode:objekttype}, rad 6. 

Eksemplet viser en mulighet på hvordan vi ønsker at dette bør fungere. Som siste parameter sender vi inn en referanse til det objekt som vi skal teste på. I den indre if-testen forsøker vi å hente opp navn til objektets instans or å undersøke objektets type. Dette er testet på mange forskjellige måter og kombinasjoner uten noe som helst resultat. 
Det er også testet med enum type som spesifiserer alle grunnleggende objekttyper som inngår i programmet (\texttt{Objekttype2.java}). 
Det var dessverre ikke mulig å bruke enum-typer for å undersøke objektinstansen via \texttt{instanceof}. Den testen i eksempelet er åpenbart feil, og om man velger teste på "<Utleier"> så går det selvfølgelig fint, men da blir den ikke så generisk som ønskelig. Ønsket var å ha en metode som fungerer uansett hva man tester på, uten å måtte spesifisere manuelt hva den skal testes mot.


\begin{lstlisting}[caption=Problem med å teste for objekttype (rad 6)., label=kode:objekttype]
    public ArrayList<T> searchForPatternIUtleier(HashSet<? extends Searchable> liste, String pattern, Object obj) {
        pattern = pattern.trim();
        pattern = pattern.toLowerCase();
        ...
                for (Searchable o : liste) {
                    if (o instanceof obj.getClass()) {
                        checkMeForResults = o.toSearch();
                        for (String s : checkMeForResults) {
                            s = s.toLowerCase();
                            if (s.contains(pattern))
                                resultList.add((T) o);
                        }
                    }
                }     
                ...
    }
\end{lstlisting}




\subsection{\texttt{Swing} og portabilitet} \label{subsec:portabilitet}
Det har vært noen utfordringer med diverse \texttt{Swing}-komponenter. Vi opplevde til stadighet at tekstfelt "<kollapset">. Det vil si at de ble bare ca 1cm bred. Neste gang man åpnet samme vindu kunne det være i orden igjen. 

En annen ting er \texttt{JEditorPane}. \texttt{HTML}-versjonen som støttes er versjon 3.2, og \texttt{CSS} 1.0. 
Det er ikke mulig å plassere komponenter slik man er vant til i dag, og bruk av tabeller ble eneste løsning. Dette er likevel noe vi gjerne skulle ha gjort bedre dersom teknologien bak \texttt{JEditorPane} hadde tillatt det\footnote{\texttt{JavaFX} har støtte for \texttt{HTML5}}.

\texttt{JPopupMenu} som vi bruker ved høyreklikking i tabellen var svært ustabil da vi brukte Java's innebygde "<Cross platform Look and Feel"> kunne popupmenyen dukke opp et god stykke unna der man høyreklikket. Dette var variabelt fra gang til gang man åpnet programmet. Etter overgang til "<Nimbus Look and Feel"> har dette, samt flere andre ting blitt betraktelig bedre.

Et problem som vi fortsatt har i Linux, men som virker å være borte i Windows, er at hver gang man høyreklikker i tabellen så må vi gjøre det to ganger for at den skal skifte fokus fra det den holdt på med. Hadde man feks hatt personobjekter i tabellen, og så søker på nytt etter boliger, så kunne man likevel få opp menyen for personobjekter ved det høyreklikket. Høyreklikket man en gang til dukket boligobjektenes meny opp.

\subsection{Genereiske metoder og klasser}
Det har også oppkommet noen utfordringer gjeldene generiske typer, metoder og klasser. Det er generelt ganske vanskelig å sette seg inn i programmering med generiske typer. Selv om man kjenner til prinsippet og hvordan det skal brukes i programmet så er utfordringen mest relatert til selve syntaksen for generiske variabler. Dette bli spesifikk vanskelig dersom det skal brukes diamantoperator sammen med eventuelle \texttt{wildcards}. Oftest må det til ganske mye testing i form av "<try and fail"> før man får frem noe som fungerer på den måte som man har tenkt seg. Under prosjektets gang har det inntruffet flere ganger at man har gitt opp en god løsning med generiske typer på grunn av at vi ikke klarte å sette sammen en kode som ble godtatt av kompilatoren, selv om løsningen i "<teorien"> burde ha fungert.