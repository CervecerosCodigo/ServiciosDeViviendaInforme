\section{Tekniske detaljer}

\subsection{Utviklingsmiljø}
Prosjektet er utviklet i NetBeans og Eclipse IDE\footnote{eng. Integrated Development Environment}. Ikoner og annen grafikk er opprettet eller editert i Gimp\footnote{The GNU Image Manipulation Program}. Generelle ikoner er opprettet og lastet ned fra \href{http://www.flaticons.net}{flaticons.net}. Innledende struktur over klasser ble tatt som UML diagram i ArgoUML program.

\subsection{Krav til programvare}
Programmet er kompilert med \texttt{javac 1.7.0\_51} og testet på tilsvarende java versjon (\texttt{OpenJDK Runtime Environment}). Programmet er grundig testet på Linux (Ubutnu 12.04 64-bit, Fedora Core 20 64-bit) samt Mac OS X. Det er gjennomført tilsvarende standard funksjonstester på MS Windows 7 og 8.1 (64-bit) for å verifisere plattformuavhengighet. 

\subsection{Versjonshåndtering}
Til versjonhåndtering brukte vi GIT via terminal og innebygd støtte i utviklingmiljøer (IDE). Lagring av prosjektet ble gjennomført sentralt via en repository på github. Repository for gruppen er privat frem til innlevering av prosjektoppgaven og kommer til å gjøres tilgjengelig for publikum etter at deadline for prosjektet har utløpt. Kildekoden vil da være tilgjengelig fra følgende linker:

\begin{description}
\item[Kildekode]
\hfill \\
\url{https://github.com/CervecerosCodigo/ServiciosDeVivienda}
\\Lagret som NetBeans Java SE prosjekt.

\item[Rapport]
\hfill \\
\url{https://github.com/CervecerosCodigo/ServiciosDeViviendaInforme}
\\ \LaTeX{} kode
\end{description}