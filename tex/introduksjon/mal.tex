\section{Mål}
Følgende mål ble satt opp ved begynnelsen av arbeidet med oppgaven:
\begin{description}
\item[Robusthet]
Det var viktig at vår kode skal være så generell så mulig slik at vi kan enkelt innføre eventuelle tillegg eller endringer i programmet som skal resultere i at kodebasen blir enkel å vedlikeholde. Koden skal derfor bygges opp med hjalp av arv og polyformisme i så stor grad som mulig. Det skal være lite bruk av parametere eller identifikatorer som låser komponenter annet enn konstanter som skal være låst.
\item[MVC]
Programmet skal være bygget opp etter MVC\footnote{eng. Model View Controller} arkitektur slik at logikk og brukergrensesnitt er helt avskjermet fra hverandre og all informasjonsutveksling blir håndtert via en kontroller. \\
Det skal være så mye gjenbruk av kode som mulig. De metoder og variabler som kan være \texttt{static} skal være det.
\item[Intuitivt brukergrensesnitt]
Brukergrensesnittet skal være enkelt å oversiktlig slik at en bruker som ikke er til fult kjent med programmet kan foreta boligsøk og sending av forespørsel. En ny megler skal rask starte opp i sin modul og på kort tid skal kunne bli kjent med programmets funksjonalitet.
\item[Faglig utfordring]
Det skal prøves å strekke strikken i forhold til å komme opp med løsninger som ikke bare løser oppgaven i henhold til pensum, men på en måte som er mest mulig slik vi tror at man ville gjort det i næringslivet. Det vil si å ikke ta snarveier, velge \texttt{JTable} foran \texttt{JList}. Bruke \texttt{MVC}, osv.
\end{description}
