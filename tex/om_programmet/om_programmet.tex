\chapter*{Om programmet}
Her følger en liste over utvalgte egenskaper ved programmet vårt.

\textbf{Strukturelle egenskaper og detaljer}

\begin{itemize}[noitemsep,nolistsep]
\item Det er brukt \texttt{MVC}-arkitektur gjennom hele programmet. Ingen vinduer kjenner til datastrukturen og omvendt. \\
\item Det har vært hovedfokus på gjenbruk av kode. Det er skrevet veldig få metoder i flere klasser som utfører samme type jobb. \\
\item Vi har laget egendefinerte \texttt{Swing}-komponenter som har utvidet funksjonalitet. Feks \texttt{CustomJTextField} med innebygget \texttt{RegEx}-tester. \\
\item Det er brukt arv i flere nivåer, spesielt i brukergrensensittet, men også i datamodellene. \\
\item Det brukes egenlagde \texttt{interface-lyttere} som muliggjør kommunikasjon mellom klasser som ellers ikke har tilknytning til hverandre. \\
\item Vi har brukt klasser som hovedpaneler i stedet for én GUI-klasse med flere "\texttt{JPanels}". Dette for å kunne bytte ut et "toppanel" eller "bunnpanel" ved senere behov, uten at hele programmet berøres. \\
\item De to arkfanene er bygget på samme \texttt{Layout}, og deler \texttt{kontrollere} og funksjonalitet. \\
\item Omfattende bruk av egendefinerte \texttt{static} konstanter som kalles opp for å sette vindustørrelser, komponentfarger osv. \\
\item Login-løsning for megler. En kan ikke komme inn i det panelet uten å logge seg på. \\
\item I det man oppretter en bolig, opprettes også en mappe på datamskinen. Om ikke bilder legges til boligobjektet så får den med et standardbilde. Andre bilder man legger til havner i denne nye mappen.\\\\
\end{itemize}


\textbf{Tekniske egenskaper og implementasjoner}

\begin{itemize}[noitemsep,nolistsep]
\item Bruk av \texttt{JTable} med egne \texttt{TabellModeller} som arver \texttt{DefaultTableModel}. Tabellen har også to forskjellige sorteringsmetoder, en \texttt{CustomCellRenderer} og automatisk kolonnebreddejustering. \\
\item Alle objekter vises i \texttt{HTML}-format i en \texttt{JEditorPane}. \\
\item Egenlaget fritekstsøk ved hjelp av å implementere et \texttt{interface} på de objektene som skal søked gjennom. \\
\item Bruk av \texttt{generiske} datatyper og metoder både ved fritekstsøket og i datastrukturen.
\item PopupMeny ved høyreklikking i tabellen. \\\\
\end{itemize}


\textbf{Egenskaper i brukergrensesnittet} \\

\begin{itemize}[noitemsep,nolistsep]
\item I det man velger et objekt i tabellen vil knappene i toppanelet aktivere/deaktivere seg ettersom hvilke valg man skal kunne ta. Man må velge en utleier før man kan opprette ny bolig. \\
\item Dobbelklikk på et objekt i tabellen vil åpne det objektets registreringsvindu med ferdig utfylt data som kan endres. \\
\item Høyreklikk på et objekt i tabellen vil gi en utvalgt popupmeny med bare de valg som gjelder for denne typen objekt. \\
\item En kan slette objekter fra tabellen ved å trykke på Delete-knappen på tastaturet. En kan ellers få dette valget i popupmenyen. \\
\item Når en har valgt et annonseobjekt så kan en klikke i "utskrifts-vinduet" for å få opp en bildeviser er en kan bla gjennom flere av annonsens bilder. \\
\item Det er tastaturkommandoer for å kunne opprette nye objekter uten å bruke mus. \\
\item En kan bruke tabulator mellom alle felter i registreringsvinduene. \\
\item En kan ikke slette boliger som er utleid \\
\item En kan ikke slette utleiere som har boliger registrert hos seg. \\
\item Alle utleiere vil i utskriftsvinduet har en liste over boliger de har registrert. \\
\item Alle Leietakere vil ha en liste over boliger de har vist interesse for. \\
\item I det en boligsøker søker på en annonse vil ikke Leietakerobjektet registreres i registeret. Først når en søknad godkjennes vil Leietakerobjekter havne i personregisteret. \\
\item Når megler godkjenner en søknad, og det ligger 4 andre søknader på samme annonse i listen, så vil de bli markert som "behandlet" og "ikke godkjent" og "grået ut". \\
\item Søknadslisten vil alltid vise ubehandlede søknader øverst, sortert innad på AnnonseID. \\
\item Tabellen er responsiv. Det vil si den oppdatere seg automatisk ved endringer av innhold. \\
\end{itemize}