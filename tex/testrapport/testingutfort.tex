\section{Testing utført}
\subsection{Testing underveis}
Vi kom sent i gang med strukturert testing av det ferdige produktet. Helt siden vi bare hadde en datastruktur har vi hatt en metode som legger inn dummydata i registerne, så de har alltid vært solide.\\
Når vi kom så langt at vi begynte å implementere programlogikk og brukergrensesnitt ble behovet for tilbakemelding om eventuelle \texttt{exceptions} og gale innstillinger en nødvendighet. 
Stort sett har vi bare brukt terminal og \texttt{System.out.println} for å fi beskjed om status på forskjellige variabler, feks før og etter en instruksjon er kjørt, eller inne i \texttt{try/catch}-blokker.

\subsection{Funksjonstest av ferdig program}
Etter overgang til \texttt{Nimbus Look and Feel} har brukeropplevelsen i Linux og Windows blitt nærmest lik. Høyreklikkmenyen fungere bedre i Windows. \\
Det er gjort to identiske tester for å teste funksjonaliteten i Linux og Windows. I Linux kjørtes programmet fra utviklermiljøet til Netbeans, mens i Windows ble det kjørt fra \texttt{jar}-fil slik som sensor vil gjøre.
