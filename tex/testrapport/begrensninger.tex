\section{Begrensninger}

\subsection{Søk} \label{subsec:begrensinger:sok}
Søkeklassen for megler (avsnitt \ref{subsec:sok:megler}, side \pageref{subsec:sok:megler}) fungerer meget bra på de datamengder som programet er testet på på sikt borde det gjennomføres en rekke forbedringer i søket. Foreløpig er det unødvendig å interere gjennom et helt set dersom brukeren ønsker tomme resultat hvilket medfører alle objekter som finnes i datalisten. Søket er også alt for generelt, det medfører at søkningen resulterer i alt for mange treffer. For eksempel det er fullt mulig å søker med en enkel bokstav, hvis brukeren skriver "<a"> i søkefeltet blir alle objekter som innholder den bokstaven i sine datafelt returnert. Eventulle forbedringer som kan legges til kan være at brukeren kan spesifisere enkelte datafelt som man ønsker å søke i tilleg i et gitt register. For eksemple dersom man ønsker å søke i person regsiter på alle som har fornavn Ola kan brukeren skrive søket med følgende "<syntaks">: \texttt{fornavn:Ola}. Slik søk vil da begrense søket til de spesifikke datafeltet.