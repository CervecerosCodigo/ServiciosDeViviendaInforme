\section{Kjente feil \texttt{(Bugs)}}
\subsection{Feil i fritekstsøk}
Det er funnet én åpenbar bug eller mangel som vi er klar over, men ikke kan fikse da det vil ødelegge datasettet vårt. Det vil si at vi kan ikke gjøre endringer i objektene for å fikse feilen uten å ødelegge datasettet med eksempeldata. 
Feilen består i at når en benytter fritekstsøket i meglerpanelet og søker på en Utleier eller Leietaker, så får man to resultater per person. 

Grunnen til dette er at klassen \texttt{Person.java} implementerer \texttt{interfacet Searchable.java} der metoden \texttt{toSearch()} definerer hva som skal være søkbart.
Det i seg selv er greit, men subklassene til Person, Leietaker, Utleier og Megler arver denne metoden og dermed returnerer de også samme objekt som superklassen. 

\textbf{FIX}: For å fikse dette midlertidig er søkerutinens datasett endret fra \texttt{ArrayList} til \texttt{HashSet}. Dette for å hindre at søkeresultatet som sendes til tabellen ikke vil inneholde mer en én referanse til samme objekt.
Permanent fix ville vært å la subklassene til \texttt{Person.java} override superklassens \texttt{toSearch()}-metode, og bare inkludere unike felt for de objektene.
