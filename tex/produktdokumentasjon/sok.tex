\section{Søk} \label{sec:sok}
Generelt i programmet finnes det to måter å søke data på: (1) \textbf{fritekstsøk} som kan foretas av megler samt (2) \textbf{annonsefilter} som foretas av boligsøker. Megleren begrenser sitt søk til et spesifikk register f.eks. boligregister eller utleierregister, deretter mottar søkeresultat baser på den tekst som er skrevet inn i tekstfeltet. Megleren kan også hente opp en full\footnote{Foreløpig er det ikke implementert en begrensing her, dersom programmet skal brukes til store datamengder borde det innføres en begrensning på antall rader som kan hentes opp til tabellen.} registerliste gjennom å bruke et tomt søkefelt eller en stjerne "<$\ast$"> som søkeparameter. 
En boligsøker søkeresultat filtrert istenden, dette gjøres gjennom å bruke kriterier som poststed, boligtype, boligstørrelse [$m^2$] og utleiepris. Dersom søkeren ikke bruker boligstørrelse eller utleiepris om parameter vil annonsene blir filtrert på kun poststed og boligtype.
Søkeklasser er plassert i \texttt{search} pakken og består av \texttt{FreeTextSearch.java} (megler) samt \texttt{AnnonseFilter.java} (boligsøker).

\subsection{Meglersøk} \label{subsec:sok:megler}
Fritekstsøk for megler består i bunn av en klasse og et interface:
\begin{description}
\item[\texttt{FreeTextSearch.java}] klassen bruker en interator for søk igjennom en generisk \texttt{HashSet} over klasser som implementerer interface \texttt{Searchable}
\item[\texttt{Searchable.java}] er et interface som er påkrevd av alle klasser som skal være mulig å bruke FreeTextSearch på. iterfacet krever at hver klasse implemetrer metoden \texttt{toSearch()} som returnerer en \texttt{String[]} over datafelt i klassen som man skal kunne søke tillempe søket på.
\end{description}

Fritekstsøk blir gjennomført på en av registrene som brukeren velger, f.eks. boligregister, utleierregister, annonseregister, mm. Søket foretas over datafelt i de objekt som inngår i et spesifikt register. For at en klasse skal kunne inngå i søket må den implementere interface \texttt{Serchable} som i sin tur krever en metode som returnerer \texttt{String[]}. Arrayen skal innholde en \texttt{toString()} representasjon av de datafelt som skal inngå i søket (eksempel \ref{kode:mgs1} og \ref{kode:mgs2} viser oppbygging interfacet og implementering av metoden \texttt{toSearch()}). 
En slik implementering gjør det enkelt å legge til søkemetoder uten å bruke mye kode som henter datafelt via klassens \texttt{get} metoder. For at klassen skal inngå i søket må den nå selv returnere sine felt i streng format.
Dersom man i fremtiden ønsker å innkludere flere datafelt i klassene som skal inngå i søket kan disse på en enkel måte implementeres i søket gjennom bruk av \texttt{Searchable} interfacet og \texttt{toSearch()} metoden.

\begin{lstlisting}[caption=Oversikt over \texttt{Searchable} interface, label=kode:mgs1]
	public interface Searchable {
   		String[] toSearch();
	}
\end{lstlisting}

\begin{lstlisting}[caption={[\texttt{toSearch()}]Implemntasjon av metode \texttt{toSearch()} i klassen \texttt{Person.java}}, label=kode:mgs2]
	@Override
    public String[] toSearch(){
        String[] searchFields = {
            String.valueOf(personID), 
            fornavn, 
            etternavn, 
            epost, 
            telefon};
        
        return searchFields;
    }
\end{lstlisting}

I eksempel \ref{kode:mgs3} presenteres selve intereringen over objektene i et generisk \texttt{HashSet} som implementerer interfacet \texttt{Searchable}. Metoden \texttt{searchForPattern()} tar inn generisk HashSet med begrensning for interface og en tekststreng som inneholder en søkeparameter. Den innkommende søkeparameteren blir renset på eventuelle blanksteg og satt til "<lowercase">. Hvis søkeparametern ikke er tom så itererer vi gjennom mengden som ble mottat via interface \texttt{Searchable}. For hver treff blir array med datafelt i form av String gjennomsøkt etter treff med metoden \texttt{contains(String s)}. 

\begin{lstlisting}[caption={[Iterasjon med \texttt{Searchable}]Iterasjon over generisk \texttt{HashSet} som implementerer interface \texttt{Searchable}}, label=kode:mgs3]
    public ArrayList<T> searchForPattern(HashSet<? extends Searchable> liste, String pattern) {

        pattern = pattern.trim();
        pattern = pattern.toLowerCase();

        if (liste != null) {
            if (pattern.equalsIgnoreCase("søk") || pattern.equals("") || pattern.equals("*")) {
                for (Searchable o : liste) {
                    resultList.add((T) o);
                }
            } else {
                for (Searchable o : liste) {
                    checkMeForResults = o.toSearch();

                    for (String s : checkMeForResults) {
                        s = s.toLowerCase();
                        if (s.contains(pattern)) {
                            resultList.add((T) o);
                        }
                    }
                }
            }
            return resultList;
        } else {
            System.out.println("En tom liste ble sendt inn til søkemetoden");
            return null;
        }
    }
\end{lstlisting}


\emph{Vi er kjent med begrensninger som finnes i denne søkemetoden, disse diskutteres i detalj i avsnitt \ref{subsec:begrensinger:sok}, side \pageref{subsec:begrensinger:sok}.}

\subsection{Annonsefilter} \label{sec:annonsesok}
I denne seksjon følger en beskrivelse av \texttt{AnnonseFilter.java}. Konstruktøren til klassen tar i mot et HashSet over annonse objekter (\texttt{HashSet<Annonse>}). Klassen internt jobber med tre HashSet som brukes mellom algoritmens filtreringstrinn, disse mengder blir initialisert i konstruktøren til klassen (se eksempel \ref{kode:annonsefilterkostn}). Metoden \texttt{annonseListeTmp} brukes mellom filtreringstrinn og \texttt{annonseListeFiltrert} blir brukt til å ferdigfiltrere resultat.

\begin{lstlisting}[caption=AnnonseFilter.java: Konstruktør, label=kode:annonsefilterkostn]
    public AnnonseFilter(HashSet<Annonse> annonseliste) {
        this.annonseListeOriginal = annonseliste;
        annonseListeFiltrert = new HashSet<>();
        annonseListeTmp = new HashSet<>();
    }
\end{lstlisting}

Uansett hvis brukeren legger inn alle filtreringsparametre eller ikke, kommer filtreringen alltid til å foregå internt etter en førdefiniert rekkefølge: (1) poststed, (2) boligtype, (3) utleiepris, (4) boligstørrelse samt (5) resterende parametre som fellesvask, hage og kjeller. 
Alle de parametrene blir sendt til filtreringmetoden fra kontroller for TopPanel i AnnonseArkFane. 
Selve interne filtreringsalgortimen foretar følgende trinn ved filtrering:
\begin{enumerate}
\item For hver element i \texttt{annonseListeOriginal} kontroller dersom \underline{poststed} stemmer og kopier elementet til \texttt{annonseListeTmp}. Etter at ha interert over alle elementer gå kopier \texttt{annonseListeTmp} til \texttt{annonseListeFiltrert}. 
\item For hvert element i \texttt{annonseListeFiltrert} kontrollere dersom \underline{boligtype} stemmer og kopier hver da elementet til \texttt{annonseListeTmp}. Etter at ha interert gjennom alle elementer i \texttt{annonseListeFiltrert} skriv over den listen med innhold fra listen \texttt{annonseListeFiltrert}. 
\item Foreta samme tilnærmingsmåte som i trinn 2 for alle søkeparametre. 
\item Returner filtrert set \texttt{HashSet<Annonse> annonseListeFiltrert}.
\end{enumerate}

I eksempel \ref{kode:filtrering2} presenteres den rekkefølge som foretas på intern filtrering. Metoden kaller opp flere interne metoder som foretar filtrering etter hver enkel parameter sendt fra kontrolleren. Eksempel \ref{kode:filtrering3} vises hvordan filtreringen foretas i en privat metode. I dette eksemplet blir data filtrert på \textit{min} og \textit{maks} pris. Filtrering i de øvrig private metodene blir foretatt på en relativt tilnærmet måte (henviser spesielt interesserte til å eventuelt betrakte løsningen i selve kildekoden til oppgaven, pakke \texttt{search}, fil \texttt{AnnonseFilter.java}). 

\begin{lstlisting}[caption={[AnnonseFilter.java: Filtreringsrekkefølge]AnnonseFilter.java: Filtreringsrekkefølge etter mottatte parametre.}, label=kode:filtrering2]
	public HashSet<Annonse> filtrerEtterParametre(String poststed, Boligtype boligtype, int prisMin, int prisMaks, int arealMin, int arealMaks, boolean harBalkong, boolean harFellesvask, boolean harHage, boolean harKjeller) {
        filtrerEtterPostSted(poststed);
        filtrerEtterBoligType(boligtype);
        filtrerEtterPrisRange(prisMin, prisMaks);
        filtrerEtterBoArealRange(arealMin, arealMaks);
        if (harBalkong) {
            filtrerEtterBalkong();
        }
        if (harFellesvask) {
            filtrerEtterFellesvaskeri();
        }
        if (harHage) {
            filtrerEtterHage();
        }
        if (harKjeller) {
            filtrerEtterKjeller();
        }
        return getFilteredResults();
    	}
\end{lstlisting}
    
\begin{lstlisting}[caption={[AnnonseFilter.java: privat filtreringsklasse]AnnonseFilter.java: Eksepel på privat filtreringsklasse etter bolig areal som foretas mellom trinn for pris og balkong.}, label=kode:filtrering3]
    private void filtrerEtterBoArealRange(int min, int maks) {
        if (min != 0 && maks != 0) {
            for (Annonse a : annonseListeFiltrert) {
                if (a.getBolig() != null) {
                    if (min <= a.getBolig().getBoAreal() && a.getBolig().getBoAreal() <= maks) {
                        annonseListeTmp.add(a);
                    }
                }
            }
            kopierTilFiltrerteResultat();
        }
    }
\end{lstlisting}