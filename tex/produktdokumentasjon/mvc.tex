\section{MVC}
\emph{Programmeringskonseptet Model View Controller} er noe vi så vidt har vært gjennom i pensum, men om helt klart var noe av det første vi bestemte oss for å ta i bruk.
Et hvert dataprogram står i fare for å ende opp med enormt store GUI-klasser om man ikke gjør noen grep. I tillegg vil all kode være eksklusiv for den klassen og ikke kunne gjenbrukes fra andre klasser. Har man to-tre vinduer som skal utføre omtrent samme type oppgaver ender man opp med å skrive omtrent identisk kode for både brukergrensesnittet og funksjonaliteten flere ganger.
Man står også i fare for veldig omstendelig kode som er vanskelig å vedlikeholde, og som gjør videre utvikling og utvidelse av funksjonalitet betraktelig vanskeligere.

\subsection{Generelt om den strukturelle oppbyggingen}
Da boligsøkere har behov for å søke etter boliger trenger de både et søkepanel, en tabell eller listevisning av søkeresultatene og en visning av de enkelt annonseobjektene.
Vi la også merke til at megler ville ha omtrent det samme behovet, bare med mulighet for å behandle boliger, personer, søknader og kontrakter i tillegg.

Dette har vi valgt å løse ved å lage to "versjoner" av samme brukergrensesnitt og kontrollerfunksjonalitet.
Det vil si en \emph{Layout} basert på \emph{BorderLayout} med et søkepanel på på toppen, en tabell på venstreside, et resultatvindu i sentrum og et knappepanel i bunnen.

Vi har dermed endt opp med en ramme av \emph{toppPanel, venstrePanel, senterPanel} og \emph{bunnPanel} som i sin tur tar inn paneler i form av egne klasser. På denne måten har vi kunnet lage to versjoner av \emph{toppPanel}, én for \emph{meglerVindu} og én for \emph{annonseVindu}. De andre panelene er felles for begge visningene, men kunne selvsagt byttes ut eller deles opp på tilsvarende måte relativt enkelt og raskt.

I de følgende delkapitlene vil det gås nærmere inn på hvordan de forskjellige programmeringslagene henger sammen og jobber seg i mellom.

\subsection{Oppstart av kontrollere og brukergrensesnitt}
Den første klassen som instansieres fra \emph{SkrivtilLesFraFil.java} under oppstart er \emph{MainController.java}.
Det er fra denne kontrolleren alt annet starter, og det er her programmet "deler" seg i tre. 
Denne kontrolleren har da det overordnede ansvar for å holde programmet sammen, og har da tilgang til alle relevante kontrollere, brukergrensesnitt og datasett. 
Ut over å starte opp brukergrensesnitter og kontrollerne finnes det noen implementasjoner av spesialsydde "\emph{lyttere}. 
Da alle våre paneler er egne javaklasser og søkeknappen i \emph{toppPanel} skal returnere et datasett til \emph{venstrePanel} sin tabell, så har vi laget \emph{interface} som kontrolleres fra \emph{MainController.java}. Skjer det en endring i \emph{toppPanel}, nærmere bestemt i \emph{ActionPerformed}-metoden, så sendes søkeresultatet via \emph{MainController.java} til \emph{venstrePanel} sine tabell-metoder. Dette helt uten at kontrollerne for de to panelene vet om hverandre.
Bruken av slike lyttere er brukt flere steder i programmet og vil bli dekket, med eksempler her \ref{sec:kontrollerlyttere} på side \pageref{sec:kontrollerlyttere}.

Når alle kontrollere er instansiert og brukergrensesnittet opprettet så kjøres det to metoder som setter opp flere \emph{lyttere} i \emph{ControllerTabell.java}, samt sende et datasett til hvert av de vinduenes tabeller.


