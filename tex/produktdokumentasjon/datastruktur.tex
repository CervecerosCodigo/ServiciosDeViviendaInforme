\section{Datastruktur \texttt{(Model)}} \label{sec:Datastruktur}

\subsection{Valg av datastruktur}
Alle dataregistrene er laget på \texttt{HashSet} fra \texttt{Collection}-rammeverket til \texttt{Java}. Vi har brukt \texttt{HashSet} for å unngå dobbellagring av like objekter. Vi har "<overridet"> \texttt{equals} og \texttt{hashcode}-metodene for objektene slik at de bare relevante datafelter brukes for å identifisere hva som er et unikt objekt og ikke. For eksempel så er det autoinkrementerte ID-feltet som gir alle objekter en unik ID, utelatt fra \texttt{equals} og \texttt{hashcode} da alle objekter da ville bli oppfattt som unike objekter.
Videre har vi ikke hatt bruk for sortering av dataene våre. Alle datasett blir sortert når de vises i tabellen.

I pakken \texttt{register} ligger selve registerne som igjen inneholder hver sitt \texttt{HashSet}. Alle registerklassene arver \texttt{Register.java} som har en del generiske metoder som fungerer på alle \texttt{HashSettene} og deres objekter.

\subsection{Datatyper}
Alle objektene, \texttt{Person, Bolig, Annonse, Søknad}, og \texttt{Kontrakt} har hver sin statiske teller som tildeles hvert objekt som opprettes. Bolig-objektene har teller fra 20000 til 29999. Annonse fra 30000 til 39999 osv.
Alle objektklassene implementerer \texttt{interfacet Searchable} som vi har utviklet selv. dette \texttt{interfacet} har en metode som heter \texttt{toString} som returnerer et \texttt{array av String-elementer}. Denne metoden definerer hva som skal inngå i fritekstsøket som beskrives mer her \ref{sec:sok} på side \pageref{sec:sok}.

\subsubsection{Bolig}
Klassen \texttt{Bolig.java} er \texttt{abstrakt} og har to subklasser; \texttt{Leilighet} og \texttt{Enebolig}. Grunnet begrensning i tid har vi ikke valgt å implementere flere typer boliger enn dette.
Boligobjektene har også en variabel \texttt{erUtleid}. Om en bolig er utleid vil ikke den kunne slettes fra datasettet \texttt{boligliste}.
Det er også en variabel som definerer en filsti til bilder av boligen. Informasjon om denne funksjonaliteten kan man lese mer om her \ref{sec:bilder}.

\subsubsection{Person}
Klassen \texttt{Person.java} er abstrakt og har som \texttt{Bolig.java} også subklasser; \texttt{Utleier.java, Leietaker.java} og \texttt{Megler.java}.
Det er ikke implementert all funksjonalitet man gjerne skulle ønske seg på disse klassene. Megler-klassen har ikke noe registreringsvindu. Det er hardkodet en megler for å kunne vise funksjonaliteten med pålogging til \texttt{meglerVindu}, men det er lagt større vekt på å få \texttt{Leietaker} og \texttt{Utleier} godt implementert da det er disse som skal behandles i annonser, søknader osv.

\subsubsection{Annonse}
Annonseobjektet består i hovedsak av et Bolig-objekt pluss noen egne felter for utleierpris, depositum, utleiers krav til leietaker osv.
Det er her også noen \texttt{Calendar}-variable som er tiltenkt bruk ved fremtidig utvidelse av programmet. Metoder \texttt{datoAnnonseRegistrert} og \texttt{datoAnnonseTasAvNett} vil kunne gi megler veldig mange muligheter for å lage statistikk på hvor lenger en annonse har vært på nett før den leies ut osv. 

\subsubsection{Søknad}
En søknad består av et annonseobjekt, leietakerobjekt, søknadsID og to \texttt{boolske} variabler for om søknaden er behandlet og godkjent.
Det er lagt opp til at mange søknader på samme annonse kan havne i megler sin liste over innkomne søknader. Søknadene sorteres da først synkende på kolonnen \texttt{erBehandlet}, og så stigende på kolonnen \texttt{annonseID}.
I det én søknad blir godkjent av megler så vil alle andre søknader automatisk bli markert som "behandlet" og som "ikke godkjent". 
Det er foreløpig ikke implementert en tilbakemeldingsfunksjon til boligsøkerne om deres søknader er godkjent eller avslått, men det ligger det selvfølgelig til rette for.

\subsubsection{Kontrakt}
En Kontrakt opprettes automatisk i det en søknad godkjennes. Annonseobjektet, Meglerobjektet og Leietakerobjektet, samt info om leiepris, depositum og dato kontrakten ble inngått er hoveddelene i kontrakten, og denne informasjonen kan ikke slettes eller endres. 
Optimalt sett skulle vi hatt en utskriftsvennlig versjon av en slik kontrakt, men det er ikke funnet tid til dette. 


