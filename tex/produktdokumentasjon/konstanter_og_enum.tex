\section{Konstanter og Enum}
Det er brukt konstakter og flere enum typer i programmet, disse finner man i \texttt{lib} pakken. Konstanter er brukt i form av publike static variabler som er tigjenegelige for tvers over hele programmet. Eksepel på dette er regex konstanter og konstanter som f.eks. b rukes til å sette opp størrelse på GUI komponenter som vinduer eller teksfelt. Enum er brukt der vi ønsker å ha mulighet til å identifisere en klassevariabel isteden for å teste f.eks. på en streng hvilket kan være ganske inneffektivt. Eksempel på dette er data i en ikke editerbar komboboks. Dersom vi fyller komboboksen med enumtyper kan vi direkte teste på hva som returneres til kontrollern fra gui isteden for å ta opp å bruke \texttt{equals()} metoden. Enum gir oss også muligheten til å eventuelt implementere metoder direkte i en enum klasse som i sin tur foreta eventuelle beregniger. 

\subsection{RegexTester.java} \label{subsec:regextest}
Denne klassen er satt sammen med hensikt å definere alle regext tester som foretas i programmet (oftest ved inhentning av data lagt inn fra bruker). Klassen består av publike regex konstanter som kan etter behov hentes opp av metoder som har innebygget regex funksjonalitet men kun behøver en definisjonsstreng med regex mønster. Et godt eksepel på dette er \texttt{CustomJTextField}. Den andre delen av programmet er boolean metoder som "<speiler"> alle regex konstanter og foretar et test på valgt regex kosntant og mottatt tekstreng som en parameter. 
I eksempel \ref{kode:regex1} presenteres en regex streng som brukes til å kontrolleren en gateadress som må begynne med en sotr bokstav, kan bestå opp til tre ord, deretter må avsluttes med et nummer som består mellom et til tre tall som kan blir etterfølgt av blankt steg og en bokstav som betegner en trappeoppgang.

\begin{lstlisting}[caption=Regexstreng for gateadresse og husnummer.,label=kode:regex1]
	public static final String GATE_ADRESSE = "^[A-ZÆØÅ]{1}[a-zæøå]{1,20}[\\s]?[A-ZÆØÅ]?[a-zæøå]*[\\s]?[A-ZÆØÅ]?[a-zæøå]*[\\s][1-9]{1}[0-9]{0,2}?[\\s]?[A-ZÆØÅ]{0,1}$";
\end{lstlisting}

Eksempel \ref{kode:regex2} presenteres privat metoden som bruker java
sin interne regex metode for å foreta test på mønsteret. Metoden blir kalt opp fra interna klasser som er spesiallaget for hver av regex konstantene (se neste eksempel).

\begin{lstlisting}[caption=Private regex test metode., label=kode:regex2]
	private static boolean patternMatchOK(String input, String regex) {
        try {
            erTestOK = input.matches(regex);
        } catch (PatternSyntaxException e) {
            System.out.println("Regex xception: input = " + input + " regex = " + regex);
        }
        return erTestOK;
  	}
\end{lstlisting}


I neste eksempel \ref{kode:regex3} vises eksepel på en av metodene som brukes til hver og enkelt definiert regex konstant. Følgende metoder brukes ofte i kontrollene som et andre trinn før et nytt objekt oprettes og leggs til i registeret (første testet blir normalt foretatt i CustomJLabel). Ettersom metoden er \texttt{static} trenger ingen man ikke initialisering av et objekt der metoden skal brukes. 


\begin{lstlisting}[caption=Static regex metode til tilhørende regex møsnter streng., label=kode:regex3]
    public static boolean testGateadresse(String gateAdresse) {
        return patternMatchOK(gateAdresse, GATE_ADRESSE);
    }
\end{lstlisting}


\subsection{Konstanter.java} \label{subsec:konstanter}
Klassen består av fleretall publike konstantanter lagret som static variabler som er tilgjengelige for alle klasser i programmet uten instansiering av klassen. I eksempel \ref{kode:konst1} presenteres noen av de konstanter som er satt opp i klassen. Eksemplet viser et kort utdrag for å presentere strukturen som brukes gjennom hele klassen.

\begin{lstlisting}[caption=Noen av static kosntanter som brukes i Konstanter klassen., label=kode:konst1]
    /**
     * Kollator rekkefølge som brukes til sortering.
     */
    public static final String KOLLATOR_REKKEFOLGE = "<\0<0<1<2<3<4<5<6<7<8<9"
            + "<A,a<B,b<C,c<D,d<E,e<F,f<G,g<H,h<I,i<J,j"
            + "<K,k<L,l<M,m<N,n<O,o<P,p<Q,q<R,r<S,s<T,t"
            + "<U,u<V,v<W,w<X,x<Y,y<Z,z<Æ,æ<Ø,ø<Å=AA,å=aa;AA,aa";

    /**
     * Felles serialiseringsnummer som brukes til unik nummer ved lagring av
     * programmets datastruktur.
     */
    public static final long SERNUM = 1234L;

    /**
     * En relativ path til alle eksterne filer som brukes i programmet som
     * serialisert data, bilder osv.
     */
    public static final String PROGRAMDATA = "programdata";

    /**
     * Serialisert fil som brukes til lagring og innlesning av all data i
     * programmet.
     */
    public static final String FILNANV = Konstanter.PROGRAMDATA + "/data.iso";
\end{lstlisting}

\subsection{GUI konstanter} \label{subsec:guikonst}
Disse er konstanter som brukes for å sette opp felles størrelse på alle komponenter i brukergrensesnittet. Disse består av to klasser:
\begin{description}

\item[\texttt{VinduStorrelse.java}]
En enum klasse som returnerer alle størrelse på vinduer som brukes i programmet. Den kan også brukes får å hente opp kun bredde eller høyde for et spesifikk vindu. Eksempel over klassens datafelt og  konstruktør presenteres i eksepel \ref{kode:guikonst1}. 

\item[\texttt{GuiSizes.java}]
En klasse med konstanter som brukes til nå sette opp interne \texttt{swing} komponenter som f.eks. bredde på \texttt{CustomJTextfield} eller \texttt{CustomJButton}.

\item[\texttt{Ikoner.java}]
Brukes til å hente opp referanser til alle ikoner som brukes tvers i hele programmet. Static konstanten blir satt opp som en \texttt{ImageIcon} variable som kan deretter brukes direkte for å sette opp et bilde i en \texttt{JPanel} eller ander komponenter. Et kort eksempel over hvordan klassen er satt opp finnes i eksepel \ref{kode:guikonst2}.
\end{description}

\begin{lstlisting}[caption=Enum klasse for vindustørrelser, label=kode:guikonst1]
public enum VinduStorrelse {

    STOR (730, 1200),
    MIDDEL (600, 800), 
    LITEN (300,400),
    TOPPANEL (150,0),
    BUNNPANEL (30,0),
    VENSTREPANEL (0,400),
    SENTERPANEL (0,0);
    
    private final int WIDTH;
    private final int HEIGHT;

    private VinduStorrelse(int HEIGHT, int WIDTH) {
        this.WIDTH = WIDTH;
        this.HEIGHT = HEIGHT;
    }
	...
}
\end{lstlisting}


\begin{lstlisting}[caption=Utsnitt fra konstantklasse med static variabler for programikoner., label=kode:guikonst2]
public class Ikoner {

    private final static String ikonerSti = new BildeFilSti().getAbsoluteGalleryPath() + "/default/ico/";
	...
    //Tabs, 24px, 4px padding, farve 606060
    public final static ImageIcon ANNONSER = new ImageIcon(ikonerSti + "Houses-24.png");
    public final static ImageIcon MEGLER = new ImageIcon(ikonerSti + "Folder-Copy-24.png");
	...
    //Applikasjonsikoner, 128px, 0px padding, E8E8E8
    public final static ImageIcon APP_ICON = new ImageIcon(ikonerSti + "boligLogo.png");
    public final static ImageIcon NY_UTLEIER = new ImageIcon(ikonerSti + "ny_utleier.png");
    public final static ImageIcon NY_BOLIG = new ImageIcon(ikonerSti + "ny_bolig.png");
	...
}
\end{lstlisting}





\subsection{Enum}
I \texttt{lib} pakken er det satt opp flere klasser av enum type, en enum klasse ble forklart i avsnitt \ref{subsec:guikonst}. De resterende enum klassene som brukes i programmet er:
\begin{description}
\item[\texttt{Boligype.java}]
Definerer de forksjellige boligtypene som: \textit{Enebolig, Tomannsbolig, Rekkehus, Leilighet, Andre}. I programmet i dagens dato brukes det kun \textit{Enebolig} og \textit{Leilighet}. Enum klassen tar dog forbehold av viderutvikling av programmet gjennom å innkludere andre boligtyper.
\item[\texttt{Sivilstatus.java}]
Brukes for populering av kombobokser ved registrering av en ny boligsøker. Enum typen blir distribuert direkte mellom kombobks og kontroller.
\item[\texttt{Arbeidsforhold.java}]
Fungerer på sammen måte som \texttt{Sivilstatus.java} og brukes til sammen funksjoner i programmet med tar for seg boligsøkerens arbeidsforhold.
\item[\texttt{Objekttype.java}]
Brukes til å definere hvilken objekttype som sendes over i flere transsakjsoner i programemt. Et eksepel på dette er renderering av tabell som blir satt etter hvilket objektytpe som blir satt via enum. Denne klassen spesifiserer objekttype på en øvre novå hvilket betyr at objektene blir spsifisert på superklasse nivå, f.eks gjør denne enum typen ingen forskjell på objekttype \texttt{Utleier} eller \texttt{Leietaker} uten kan kun vise at objektet som sendes over er av type \texttt{Person}.
\item[\texttt{Objekttype2.java}]
Fungerer og brukes på samme måte som \texttt{Objekttype.java} med innholder detaljert informasjon over hvilke objekter som kan passeres mellom transaksjonene. Her gjør vi altså forskjell mellom underliggende klasser som \texttt{Utleier} og \texttt{Leietaker}.

\end{description}