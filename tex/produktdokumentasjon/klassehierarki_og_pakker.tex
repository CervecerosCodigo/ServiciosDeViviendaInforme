\section{Klassehierarki og pakker}
Da implementasjonen vi har lagt oss på krever mange javafiler, som ofte gjerne har relativt lite innhold har vi også delt javaklassene inn i flere pakker under mappen “src”.

\subsection*{Pakken controller}
Controller og undermappen Registrer inneholder all funksjonalitet til alle JPanels, deres innhold og registreringsvinduer. Funksjonaliteten og nærmere beskrivelse av disse klassene finnes i andre deler av dokumentasjonen.

\subsection*{Pakken lib}
Lib er biblioteket av statiske metoder, konstanter og enum, samt andre klasser og metoder som ikke naturlig hører til MVC-tankegangen.
Her finnes også våres RegEx-konstanter som er brukt på alle Tekstfelt i GUI, samt på nytt i kontrollerne. Det testes altså to ganger, før registrering og validering før objektet opprettes/endres.

\subsection*{Pakkenne model og register}
Model inneholder klassene det lages objekter av, som igjen legges inn i datasettene. Klassene Person og Bolig er abstrakte med hver sine subklasser. Klassen TabellModell er en abstrakt klasse som arver DefaultTableModel, og den har igjen klasser som arver dens funksjonalitet. Dette vil en kunne lese mer om i beskrivelsen av tabellimplementasjonen.
Pakken Register inneholder ganske tomme klasser. Hensikten med å ha egne registerklasser, ut over det å ha alle Hashsett i MainController er å implementere serializable. Det er altså disse klassene som serialiseres, og som inneholder alle datasett som blir skrevet til og lest fra fil.

\subsection*{Pakken search}
Search inneholder filene for søkeimplementasjonen vi har utviklet. Dette er en helt egenutviklet løsning som vil bli beskrevet nærmere i avsnittet for søkeimplementasjonen \ref{sec:sok}, side \pageref{sec:sok}.

\subsection*{Pakken serviciosdevivienda}
Det er to klasser her. Mainmetoden vår, serviciosdevivienda.java, samt SkrivTilLesFraFil.java som utfører all serialisering av data. Serialisering blir dokumentert nærmere her \ref{sec:serialisering}, side \pageref{sec:serialisering}.

\subsection*{Pakken view og undermappen registrer}
Disse to pakkene inneholder alle brukergrensesnitt, egendefinerte javakomponenter som \emph{CustomJTextField, CustomSubPanel} osv.
Det er brukt en stor mengde kreativitet og tankevirksomhet for å komme frem til løsningen vi har endt opp med, og det vil blir beskrevet flere andre steder i rapporten.


