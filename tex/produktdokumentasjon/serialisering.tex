\section{Serialisering} \label{sec:serialisering}
Skriving og innlesing av data blir kalt opp fra programmets \texttt{main} metode. Les og skrive metoder finnes i \texttt{SkrivTilLesFraFil.java}. Den klassen initierer \texttt{MainController.java} som er programmet primære kontroller i MVC arkitekturen. Alle registre blir initialisert fra les/skrive klassen med hensikt å ha mulighet til å seriasliere alle registrene på plass. Ved innlesing av lagret data blir alle registrene passert som komponenter til \texttt{MainKontroller}, se eksempel \ref{kode:ser2}. Eksempel \ref{kode:ser1} viser hvordan registrene og static variabler blir håndtert ved skrivning og serialisering. Programmets arkitektur baserer seg på tellevariabler som holder kontroll på antall objekter som til hver gang blir opprettet og lagt til i hvert enkelt register. Statiske filer blir skrevet i spesifikk rekkefølge etter at alle registrene (\texttt{HashSet}) er skrevet til fil.

Lesing og skriving er opprettet etter \texttt{Java SE 6} struktur, derfor benytter vi oss ikke av  "<try-med resurser"> tilnærming. Hensikt med den brukte løsningen er at det oppleves som at vi får en mer oversiktlig kode.


\begin{lstlisting}[caption=Serialisering og skriving av data.,label=kode:ser1]
    public void lagreData() {
        try {
            FileOutputStream fos = new FileOutputStream(new File(Konstanter.FILNANV));
            ObjectOutputStream out = new ObjectOutputStream(fos);

            out.writeObject(personliste);
            out.writeObject(boligliste);
            out.writeObject(annonseliste);
            out.writeObject(kontraktliste);
            out.writeObject(soknadsliste);

            out.writeInt(Person.getTeller());
            out.writeInt(Bolig.getTeller());
            out.writeInt(Annonse.getTeller());
            out.writeInt(Soknad.getTeller());
            out.writeInt(Kontrakt.getTeller());

            out.close();

        } catch (IOException e) {
            System.out.println(e.fillInStackTrace());
        }
    }
\end{lstlisting}

\begin{lstlisting}[caption=Innlesing av serialisert data.,label=kode:ser2]
    public void lesInnData() {
        try {
            FileInputStream fis = new FileInputStream(new File(Konstanter.FILNANV));
            ObjectInputStream in = new ObjectInputStream(fis);

            personliste = (HashSet<Person>) in.readObject();
            boligliste = (HashSet<Bolig>) in.readObject();
            annonseliste = (HashSet<Annonse>) in.readObject();
            kontraktliste = (HashSet<Kontrakt>) in.readObject();
            soknadsliste = (HashSet<Soknad>) in.readObject();

            Person.setTeller(in.readInt());
            Bolig.setTeller(in.readInt());
            Annonse.setTeller(in.readInt());
            Soknad.setTeller(in.readInt());
            Kontrakt.setTeller(in.readInt());

            in.close();
            mainController = new MainController(personliste, boligliste, annonseliste, kontraktliste, soknadsliste);

        } catch (IOException e) {
            System.out.println(e.fillInStackTrace());

        } catch (ClassNotFoundException e) {
            System.out.println(e.fillInStackTrace());
        }
    }
\end{lstlisting}